\section{Grounding referring expressions}
\label{sec:preexperiments}

This chapter serves two purposes.
First, the generated dataset from the precious section is validated.
For this, models are trained to both generate referring expressions of the target object and understand existing referring expressions.
A success of these experiments indicates that the target objects in the datasets are possible to refer to and the datasets can be used in more complex setups in language games.

Secondly, the experiments in this chapter provide the basis for the setup of the agents in the language games.
Language games are very complex setups for machine learning models.
The models need to solve multiple tasks at the same time in order to solve the overall problem.
For instance, in a simple setup of a game two agents are involved.
The first agent, the sender, is shown a scene with objects and needs to communicate one target object to the other agent, the receiver.
The receiver is shown the same scene and needs to identify the target object with respect to the message of the sender.
In this case, the sender first needs to learn to encode the scene, all objects and their attributes, as well as the information about the target object into its own game specific space.
In a next step it needs to learn how to translate this encoding into a message that is sent to the receiver.
The receiver then needs to learn to decode this message, after which it needs to learn how to combine the decoded message with its own encoding of the scene and objects.
And finally it needs to learn how to identify the target object with this information.
There are many points of possible failure to train the agents.
% SD: This should go at the beginning where language games with neural agents are introduced.
% DK: TODO

For this reason, we decided to divide the main problem and let the models learn simpler subtasks and increase the complexity step by step.\footnote{\href{https://github.com/DominikKuenkele/MLT\_Master-Thesis}{https://github.com/DominikKuenkele/MLT\_Master-Thesis}}
This will give a very detailed overview where the models struggle to learn and in which ways they can be improved.
Mainly, the tasks are separated into language games with two agents and classical machine learning tasks without any communication, namely only one 'agent' that solves the task alone.
With this division, we can analyze the learning of the encodings of the scenes separately from the learning of producing and decoding messages.

% TODO maybe move to front...
The final objective of this thesis is to find out, how agents can communicate about relations of objects based on their attributes, namely how to generate and understand referring expressions.
Because of that, the first experiments focus on extracting information from images and combining them with structured knowledge about the objects.
Here, we structured the experiments into three levels.
In the first level, the models are trained to learn the position of objects in the image and attend to specific regions of the image, by understanding a referring expression of the target object.
% SD: This is a natural language understanding or reference resolution task.
% DK: TODO maybe put into methods...
In the second level, the models are trained to differentiate objects in the scene from each other, again by understanding referring expressions.
% SD: Object identification task
% DK: TODO
In the last level, the models are trained to generate referring expressions, more specifically the models learn to caption and describe objects in the image.
% SD: Referring expression generation task
% DK: TODO
These combined experiments should lay the basis for how to build up the agents in the language games.
% SD: We are validating the dataset on 3 tasks
% DK: done

For all experiments in this section, the batch size is 32 samples and the optimizer is \emph{Adam} \citep{Kingma2015}.
8000 randomly selected samples are used for training, the remaining 2000 samples for testing.
Other hyperparameters are specified for each experiment.